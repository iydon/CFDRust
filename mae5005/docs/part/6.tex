\section{第六次作业}

\begin{homework}[label={H:6-1}]
    Consider integrating the following equation in time, for $u(t)$,
    \begin{equation}\label{E:6-1-1}
        \dv{u}{t} = F(u)
    \end{equation}
    using the idea of the third-order Runge-Kutta method, namely,
    \begin{equation*}
        \begin{aligned}
            u^{(1)} &= u^n + \dd t F(u^n) \\
            u^{(2)} &= u^n + \frac{\dd t}{2} \qty[\alpha F(u^n)+(1-\alpha)F(u^{(1)})] \\
            u^{(3)} &= u^n + \dd t \qty[\beta F(u^n)+\gamma F(u^{(2)})+(1-\beta-\gamma)F(u^{(1)})]
        \end{aligned}
    \end{equation*}
    where $F(u)$ is an arbitary function of $u$. Note in this case $u^{(1)}$ can be viewed as the estimate of $u$ at $t^{n+1}$, $u^{(2)}$ can be viewed as the estimate of $u$ at $t^{n+0.5}$.

    \begin{enumerate}[label=(\alph*)]
        \item Using Taylor expansions, derive all constraints that you need to ensure that the above method is third-order accurate, meaning the truncation error, relative to the ODE~\eqref{E:6-1-1}, is of the order $\order{\dd t^3}$. In particular, show that
            \begin{equation*}
                \begin{aligned}
                    \gamma+2(1-\beta-\gamma) &= 1 \\
                    \frac{\gamma}{8}+\frac{1-\beta-\gamma}{2} &= \frac{1}{6} \\
                    \frac{\gamma}{2}(1-\alpha) &= \frac{1}{6}
                \end{aligned}
            \end{equation*}
        \item Then solve the above linear system to show that $\alpha=1/2$, $\beta=1/6$, $\gamma=2/3$. This resulting scheme is well known as the third-order TVD Runge-Kutta scheme.
    \end{enumerate}
\end{homework}

pass



\begin{homework}[label={H:6-2}]
    Consider the 1D unsteady advection problem
    \begin{equation}\label{E:6-2-1}
        \pdv{u}{t} + a\pdv{u}{x}=0,
        \quad
        \text{where $-2\leq x\leq 4$},
    \end{equation}
    with $a=1$ and the initial condition
    \begin{equation}\label{E:6-2-2}
        u(x, t=0) = \begin{cases}
            2.0 & \text{if $x<0$} \\
            0.0 & \text{if $x>0$}
        \end{cases}
    \end{equation}

    In this case the analytical solution is known, at least for early times when the front has not reached the right boundary. You can assume a zero-gradient boundary condition at both ends of the domain. Now consider the following scheme provided in the lecture, based on the cell-wise linear reconstruction
    \begin{equation}\label{E:6-2-3}
        u^{n+1}_j = u^n_j - \frac{a\Delta t}{\Delta x}\qty(u^n_j-u^n_{j-1}) - \frac{a\Delta t}{2\Delta x}\qty(\Delta x-a\Delta t)\qty(\sigma^n_j-\sigma^n_{j-1})
    \end{equation}
    where $\sigma^n_j$ is a properly defined slope for cell $j$. Assume $\Delta x=0.1$ and $\Delta t=0.05$, and use 60 grid points located at $x_i=(i+0.5)\Delta x$, for $i=-20, -10, \ldots, -1, 0, 1, \ldots, 38, 39$.

    \begin{enumerate}[label=(\alph*)]
        \item Obtain the numerical solutions by hand at $t=0.05$ (only one time step) for three cases: the downwind slope (i.e., the Lax-Wendroff scheme), the van Leer limiter (form 2 in the lecture), and the SUPERBEE limiter. You may use these data to validate your codes in Part (b).
        \item Write three Fortran Codes using, respectively, the downwind slope (i.e., the Lax-Wendroff scheme), the van Leer limiter (form 2 in the lecture), and the SUPERBEE  limiter. Plot all solutions (three numerical solutions plus the analytical solution) together at $t=0.05$ (one time step), $0.2$ (4 time steps), $1.0$ (20 time steps), and $2.0$ (40 time steps), respectively (one plot for each time). Hint: For easy coding, you can add virtual nodes on the two ends of the domain. For the Lax-Wendroff scheme, adding one virtual node at each end is sufficient, while for the other two, adding two virtual nodes at each end will be adequate.
        \item Comment on the results from the above three treatments. Also plot and compare the total variation (TV) as a function of time for the three cases (plot all TV data on a single figure). The total variation is defined as
            \begin{equation}\label{E:6-2-4}
                TV = \sum_{\forall i}\abs{u_{i+1}-u_i}.
            \end{equation}
    \end{enumerate}
\end{homework}

pass



\begin{homework}[label={H:6-3}]
    Consider the local $P^2$ reconstruction $u(\zeta)=a+b\zeta+c\zeta^2$ of $u(x)$ in terms of the three cell-averaged values $\bar{u}_{j-1}$, $\bar{u}_j$, $\bar{u}_{j+1}$ using the cells $j-1$, $j$, $j+1$.

    \begin{enumerate}[label=(\alph*)]
        \item Using the local coordinate $\zeta=x-x_j$, determine $a$, $b$, $c$.
        \item Compute the local values at the three node points $x_{j-1}$, $x_j$, $x_{j+1}$, in terms of $\bar{u}_{j-1}$, $\bar{u}_j$, $\bar{u}_{j+1}$.
        \item Show the following specific local reconstruction relation
            \begin{equation}\label{E:6-3-1}
                u_{j+\frac{3}{2}} = \frac{1}{3}\bar{u}_{j-1} - \frac{7}{6}\bar{u}_j + \frac{11}{6}\bar{u}_{j+1}.
            \end{equation}
        \item As a comparison, show that the usual (Lagrangian) pointwise interpolation relation is instead
            \begin{equation}\label{E:6-3-2}
                u^L_{j+\frac{3}{2}} = \frac{3}{8}u_{j-1} - \frac{5}{4}u_j + \frac{15}{8}u_{j+1}.
            \end{equation}
    \end{enumerate}

    Observations: Therefore, the reconstruction relation and interpolation relation are close [the coefficients in (2) are $0.333, −1.167, 1.833$; and coefficients in (3) are $0.375, −1.250, 1.875$] but are not the same. However, in both cases, the coefficients are summed to one (a basic requirement when applied to a constant function).
\end{homework}

pass
