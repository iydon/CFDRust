\section{第二次作业}

\begin{homework}[label={H:3-1}]
    In the lecture, we briefly mentioned the steady-state heat conduction problem in a square domain, with temperature on the three sides being zero, and the one remaining side at $T_0$ ($T_0$ is a positive constant temperature). The problem can be described as follows:
    \begin{equation}\label{E:3-1-1}
        \pdv[2]{T}{x} + \pdv[2]{T}{y} = 0,
    \end{equation}
    with the boundary conditions
    \begin{equation}\label{E:3-1-2}
        T(x, 0) = T_0, \quad
        T(x, L) =   0, \quad
        T(0, y) =   0, \quad
        T(L, y) =   0.
    \end{equation}
    By the method of separation of variables, the analytical solution can be expressed as
    \begin{equation}\label{E:3-1-3}
        T(x, y) = \sum_{n=1}^\infty C_n\sin\qty(\frac{n\pi x}{L})\sinh\qty(\frac{n\pi (L-y)}{L})
    \end{equation}

    \begin{enumerate}[label=(\alph*)]
        \item Show that the form given by Equation~\eqref{E:3-1-3} satisfies the governing differential equation, Equation~\eqref{E:3-1-1}.
        \item Show that the form given by Equation~\eqref{E:3-1-3} satisfies all three zero-temperature boundary conditions.
        \item Determine the coefficients $C_n$, explicitly, using the boundary condition $T(x, 0)=T_0$.
        \item Write a Fortran code to obtain $T/T_0$ as a function of $x/L$ and $y/L$, then make a contour plot of the temperature field in the domain. Also plot the temperature profiles at the two center lines, $x/L=0.5$ and $y/L=0.5$.
        \item Compute the net heat flux through each of the four sides, and show that the sum of the heat flux through four sides is zero.
    \end{enumerate}
\end{homework}

pass



\begin{homework}[label={H:3-2}]
    Show, by Taylor expansion, that
    \begin{equation}\label{E:3-2-1}
        \dv[2]{f(x_i)}{x} = \frac{f_{i+1}-2f_i+f_{i-1}}{\Delta x^2} - \frac{\Delta x^2}{12}\dv[4]{f(x_i)}{x} + \order{\Delta x^4}
    \end{equation}
    For $f(x)=x^5$,  what is the error in computing $\dv*[2]{f}{x}$ at $x_i=1$ with $\Delta x=0.1$? How 
    is the error compared to the leading-order estimate in Equation~\eqref{E:3-2-1}?
\end{homework}

pass



\begin{homework}[label={H:3-3}]
    Consider the following finite-difference representation for the cross partial derivative,
    \begin{equation}\label{E:3-3-1}
        \pdv[2]{u(i, j)}{x}{y} \approx \frac{u_{i+1, j+1} - u_{i-1, j+1} - u_{i+1, j-1} + u_{i-1, j-1}}{4\Delta x\Delta y}
    \end{equation}

    \begin{enumerate}[label=(\alph*)]
        \item What is the exact expression of the leading-order error term?
        \item Now it one applies the above approximation to the location $x=0$ and $y=1$, with $\Delta x=\Delta y=0.1$, for a field $u(x, y)=y^2\sin x$, compute the exact value of the cross partial derivative and the value from the finite-difference approximation.
        \item Confirm that the difference between the exact value and the value from the finite-difference approximation is indeed very similar to the leading-order error term you obtained in (a).
    \end{enumerate}
\end{homework}

pass



\begin{homework}[label={H:3-4}]
    Using Taylor expansion, derive the leading-order truncation error term for the following Crank-Nicholson time integration scheme
    \begin{equation}\label{E:3-4-1}
        \frac{u(t_{n+1})-u(t_n)}{\Delta t} = \frac{1}{2}F(u(t_n), t_n) + \frac{1}{2}F(u(t_{n+1}), t_{n+1})
    \end{equation}
    for solving the ordinary differential equation $\dv*{u}{t}=F(u(t), t)$.
\end{homework}

pass
